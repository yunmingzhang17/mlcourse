\iffalse

INSTRUCTIONS:

  Clip out the ********* INSERT HERE ********* bits below and insert
appropriate TeX code.  Once you are done with your file, run

  ``latex sample.tex''

from the Athena shell.  If your TeX code is clean, the latex will exit
back to a prompt.  Once this is done, run

  ``dvips -t letter -f sample.dvi > sample.ps''

You can then convert the PostScript output to PDF:
  ``ps2pdf sample.ps''

\fi
\documentclass[11pt]{article}
\usepackage{amsfonts}
\usepackage{latexsym}
\setlength{\oddsidemargin}{.25in}
\setlength{\evensidemargin}{.25in}
\setlength{\textwidth}{6in}
\setlength{\topmargin}{-0.4in}
\setlength{\textheight}{8.5in}

\newcommand{\handout}[5]{
   \renewcommand{\thepage}{#1-\arabic{page}}
   \noindent
   \begin{center}
   \framebox{
      \vbox{
    \hbox to 5.78in { {\bf 6.046: Intro to Algorithms} \hfill Due: #2 }
       \vspace{4mm}
       \hbox to 5.78in { {\Large \hfill #1: #5  \hfill} }
       \vspace{2mm}
       \hbox to 5.78in { {\it Instructor: #3 \hfill TA: #4} }
      }
   }
   \end{center}
   \vspace*{4mm}
}

\newcommand{\al}{\alpha}
\newcommand{\Z}{\mathbb Z}

\begin{document}

\handout{Problem Set N}{Due Date here}{Instructor Here}
{Your T.A. Here}{Your Name Here}

I collaborated with Collaborators Here.

\section*{Problem 1}

\subsection*{A}
Here are some examples of using LaTeX symbols:

$\Theta(n^2) = \sum_{i=0}^n i$

$\Theta(n) = \Omega(n)$

$\Theta(n) \neq \omega(n)$

\subsection*{B}
\begin{itemize}
\item You can make
\item bullet
\item lists.
\end{itemize}


\begin{enumerate}
\item Or
\item numbered
\item lists.
\end{enumerate}



\section*{Problem 2}

this is a separate tex file
You can also do more complex math formulas:
\[\Theta(n^2) = \sum_{i=0}^n i\]


\end{document}


